Nachfolgend wird die Größe der "Rohdaten" für eine 1-tps-Datenbank bzw. allgemein für eine n-tps-Datenbank bestimmt.
Die betrachtete Datenbank hat vier initialisierte Relationen (Accounts, Branches, Tellers und History).
Diese werden durch einen vorgegebenen \textbf{Skalierungsfaktor n} in der n-tps-Datenbank skaliert. Also je nach
Angabe werden die Datenbankeinträge, sowie die Größe der Datenbank erweitert oder verringert. Die einzelnen Skalierungsfaktoren
der verschiedenen Relationen zur n betragen einmal bei \textbf{Accounts (n * 100.000) Tupel}, bei \textbf{Braches n Tupel},
bei \textbf{Tellers (n * 10) Tupel} und \textbf{0 Tupel bei History}. Diese werden hiernach in einer Tabelle mit den
verschiedenen Relationen und deren Tupelanzahl aufgezeigt
\begin{table}[h!]
    \centering
    \begin{tabular}{|p{2.5cm}|c|c|c|c|}
        \hline
        Relation & Accounts & Branches & Tellers & Histroy \\  \hline
        n = 1 & 100.000 & 1 & 10 & 0 \\ \hline
        n = 2 & 200.000 & 2 & 20 & 0 \\ \hline
        n = 5 & 500.000 & 5 & 50 & 0 \\ \hline
        n = 10 & 1.000.000 & 10 & 100 & 0 \\ \hline
        n = 20 & 2.000.000& 20 & 200 & 0 \\ \hline
        n = 50 & 5.000.000& 50 & 500 & 0 \\ \hline
    \end{tabular}
    \caption{Tabelle für die Größe der Rohdaten in der n-tps-Datenbank}
    \label{tab:1}
\end{table}