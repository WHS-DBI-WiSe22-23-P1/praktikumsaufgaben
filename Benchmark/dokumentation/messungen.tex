Anschließend werden die Messergebnisse der Laufzeit bei den verschiedenen Optimierung des Codes, sowie der Datenbankmanagement-Konfigurationen gezeigt.

\subsection{Server}\label{subsec:server}
Zunächst die Messergebnisse bei der Erstellung und Verarbeitung der Anfragen lokal auf dem DBMS mit Docker-Einbindung.
\begin{table}[h!]
    \centering
    \begin{tabular}{|c|c|c|}
        \hline
        DB 1 (n = 1) [s] & DB 2 (n = 2) [s] & Bemerkungen \\  \hline
        1749 & 0 & n = 1: 100.000 Tupel \\ \hline
    \end{tabular}
    \caption{Tabelle für Messergebnisse in Sekunden: Lokal - 1. Version}
    \label{tab:5}
\end{table} \\

\begin{table}[h!]
    \centering
    \begin{tabular}{|c|c|c|}
        \hline
        DB 1 (n = 1) [s] & Bemerkungen \\  \hline
        1321 & 10 Tupel pro Batch \\ \hline
        1501 & 100 Tupel pro Batch \\ \hline
        1352 & 1.000 Tupel pro Batch \\ \hline
        1544 & 10.000 Tupel pro Batch \\ \hline
        1608 & 100.000 Tupel pro Batch \\ \hline
    \end{tabular}
    \caption{Tabelle für Messergebnisse in Sekunden: Lokal - 2.1 Version (Optimierung: Batch - Bündeln der Statements)}
    \label{tab:6}
\end{table}

\begin{table}[h!]
    \centering
    \begin{tabular}{|c|c|c|}
        \hline
        DB 1 (n = 1) [s] & Bemerkungen \\  \hline
        1943 & 1 Tupel pro Batch \\ \hline
        308 & 10 Tupel pro Batch \\ \hline
        143 & 100 Tupel pro Batch \\ \hline
        150 & 1.000 Tupel pro Batch \\ \hline
        153 & 10.000 Tupel pro Batch \\ \hline
        135 & 100.000 Tupel pro Batch \\ \hline
    \end{tabular}
    \caption{Tabelle für Messergebnisse in Sekunden: Lokal - 2.2 Version (Optimierung: Batch - Bündeln der Statements + ohne AutoCommit)}
    \label{tab:7}
\end{table}

\begin{table}[h!]
    \centering
    \begin{tabular}{|c|c|c|c|}
        \hline
        DB 1 (n = 10) [s] & DB 2 (n = 20) [s] & DB 3 (n = 50) [s] & Bemerkungen \\  \hline
        44 & 84 & 295 & Multithreading (Java Async Await)\\ \hline
    \end{tabular}
    \caption{Tabelle für Messergebnisse in Sekunden: Lokal - 3. Version (Optimierung: Multithreading - asynchrone Ausführung)}
    \label{tab:8}
\end{table}

\newpage
\subsection{Remote}\label{subsec:remote}
Hierauf die Messergebnisse bei der lokalen Erstellung und externen Verarbeitung der Anfragen auf dem VM-DBMS. \\
\begin{table}[h!]
    \centering
    \begin{tabular}{|c|c|c|}
        \hline
        DB 1 (n = 1) [s] & DBI 2 (n = 2) [s] & Bemerkungen \\  \hline
        466 & 791 & n = 1: 100.000 Tupel / n = 2: 200.000 Tupel\\ \hline
    \end{tabular}
    \caption{Tabelle für Messergebnisse in Sekunden: Remote - 1. Version}
    \label{tab:9}
\end{table} \\

\begin{table}[h!]
    \centering
    \begin{tabular}{|c|c|c|c|}
        \hline
        DB 1 (n = 10) [s] & DB 2 (n = 20) [s] & DB 3 (n = 50) [s] & Bemerkungen \\  \hline
        32 & 61 & 195 & Multithreading (Java Async Await)\\ \hline
    \end{tabular}
    \caption{Tabelle für Messergebnisse in Sekunden: Remote - 3. Version (Optimierung: Multithreading - asynchrone Ausführung)}
    \label{tab:10}
\end{table}