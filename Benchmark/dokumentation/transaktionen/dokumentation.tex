Anschließend werden die unterschiedlichen Versionen des Java-Programms im Hinblick auf die verschiedenen Lasttransaktionen dokumentiert und anhand von Versionen gegliedert.
Der gesamte Quellcode des Programms ist im Anhang zu finden.

\subsection{Version 1}\label{subsec:version1}
In der ersten Version wurde ein rudimentäres Programm erstellt, das die unterschiedlichen Lasttransaktionen einfach und ohne Verbesserungen ausführt.
Hierbei wird bei der Kontostandtransaktion ein SELECT-Statement ausgeführt, das den aktuellen Kontostand (balance) eines Kontos mit der Kontonummer (accid) aus der Relation accounts zurückgibt
\lstinputlisting[breaklines,language=Java,label={lst:version1-getBalanceFromAccount},caption={Quellcode zur Kontostandstransaktion der ersten Version}]{assets/code/version1/getBalanceFromAccount.java}
Die Einzahlungstransaktion unterteilt sich in mehrere Einzelaktionen.
In den zuerst der Kontostand (balance) aus der Relationen accounts mit dem Einzahlungsbetrag (delta) aktualisiert wird und dann die Beträge (balance) von den jeweiligen Geldautomaten (teller) mit der Nummer (tellerid) und der jeweiligen Zeigstelle (branch) mit der Nummer (branchid) mittels der Methode updateBalance mit dem Einzahlungsbetrag (delta) aktualisiert werden.
Als letztes wird dann noch ein Einzahlungsbeleg in der Relation history hinterlegt.
\lstinputlisting[breaklines,language=Java,label={lst:version1-updateBalance},caption={Quellcode zur Einzahlungstransaktion der ersten Version}]{assets/code/version1/updateBalance.java}
\lstinputlisting[breaklines,language=Java,label={lst:version1-insertHistory},caption={Quellcode zur Einfügetransaktion vom Einzahlungsbeleg}]{assets/code/version1/insertHistory.java}
Die letzte Transaktion gibt die Anzahl der aktualisierten Kontostandsbeträge (accbalance) in der Relation history mittels eines SELECT-Statements zurück.
In dem alle Tupel mit dem genauen Kontostandsbetrag (delta) gezählt werden.
\lstinputlisting[breaklines,language=Java,label={lst:version1-getNumberOfDeltaBalance},caption={Quellcode zur Analysetransaktion der ersten Version}]{assets/code/version1/getNumberOfDeltaBalance.java}

\subsection{Version 2}\label{subsec:version2}
In der zweiten Version wurden die beiden Transaktionen Kontostand-TX und Analyse-TX mit der Verbesserung Database Views geändert.
Analog zu ersten Version wurden hierbei nur die SELECT-Statements (Zeile 2 (getBalanceFromAccountWithView / getNumberOfDeltaBalanceWithViews)) geändert, sodass jetzt durch die Views die Anfragen bearbeitet werden.
Weitere Einzelheiten können unter \hyperref[sec:optimierungen]{\textbf{Optimierungen}} gefunden werden.
\lstinputlisting[breaklines,language=Java,label={lst:version2-getBalanceFromAccount},caption={Quellcode zur Kontostandstransaktion der zweiten Version}]{assets/code/version2/getBalanceFromAccountWithViews.java}
\lstinputlisting[breaklines,language=Java,label={lst:version2-getNumberOfDeltaBalance},caption={Quellcode zur Analysetransaktion der zweiten Version}]{assets/code/version2/getNumberOfDeltaBalanceWithViews.java}

\subsection{Version 3}\label{subsec:version3}
In der dritten Version wurden die Prepared-Statements aus der ersten Version verbessert, indem diese nur einmal während des Startens erstellt und dann an das DBMS übermittelt werden.
Dementsprechend werden diese nicht für jede einzelne Kontostand-TX oder Analyse-TX erstellt und gesendet, sondern es wird nur noch der Parameter (accid / accbalance) geändert und dann das Statement ausgeführt.
\lstinputlisting[breaklines,language=Java,label={lst:version3-initialisePreparedStatements},caption={Quellcode zur Initialisierung der Prepared-Statements der dritten Version}]{assets/code/version3/initialisePreparedStatements.java}
\lstinputlisting[breaklines,language=Java,label={lst:version3-getBalanceFromAccountWithInitialisedStatements},caption={Quellcode zur Kontostandstransaktion der dritten Version}]{assets/code/version3/getBalanceFromAccountWithInitialisedStatements.java}
\lstinputlisting[breaklines,language=Java,label={lst:version3-getNumberOfDeltaBalanceWithInitialisedStatements},caption={Quellcode zur Analysetransaktion der dritten Version}]{assets/code/version3/getNumberOfDeltaBalanceWithInitialisedStatements.java}

\subsection{Version 4}\label{subsec:version4}
In der vierten Version wurde nun das Indexing in der Tabelle history verbessert.
Weitere Informationen im Abschnitt \hyperref[sec:optimierungen]{\textbf{Optimierungen-Indices}}.

\subsection{Version 5}\label{subsec:version5}
In der fünften Version wurde an den nötigen Stellen ein manueller Commit ausgeführt.
Damit ist das Paket der Datenkonsistenz größer und es wird Leistung eingespart.
\lstinputlisting[breaklines,language=Java,label={lst:version5-updateBalance},caption={Quellcode mit manuellen Commits}]{assets/code/version5/updateBalance.java}
