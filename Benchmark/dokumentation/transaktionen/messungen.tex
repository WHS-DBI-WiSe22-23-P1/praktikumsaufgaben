\subsection{Maximaler erreichbarer Transaktionsdurchsatz}\label{subsec:maximaler-erreichbarer-transaktionsdurchsatz}
Um den maximalen erreichbaren Transaktionsdurchsatz zu berechnen, müssen nur die begrenzenden Variablen, in diesem Fall Think-Time mit 50 ms, angeschaut werden.
Infolgedessen wird theoretisch angenommen, dass die verschiedenen Lasttransaktionen bei der Ausführung keine Zeit beanspruchen.
Schließlich ergibt sich dann aus dem Messzeitraum von 300s (5 min) und der begrenzenden Variable mit 0,05s der maximale erreichbare Transaktionsdurchsatz pro Thread im Loaddriver von 6.000 TXs (300s / 0,05s) und somit 20 TXs pro Sekunde.
Außerdem kann aus der relativen Gewichtung für die unterschiedlichen Lasttransaktionen folgende Verteilung der Transaktionsanzahl der verschiedenen Lasttransaktionen und Last-Profilen bestimmt werden.
\begin{table}[h!]
    \centering
    \begin{tabular}{|c|c|c|}
        \hline
        Lasttransaktion & relative Gewichtung [\%] & Transaktionsanzahl [TXs] \\  \hline
        Kontostands-TX & 35 & 2.100 \\ \hline
        Einzahlungs-TX & 50 & 3.000 \\ \hline
        Analyse-TX & 15 & 900 \\ \hline
    \end{tabular}
    \caption{Tabelle für die Verteilung der Transaktionsanzahl der unterschiedlichen Lasttransaktionen bei verschiedenen relativen Gewichtungen}
    \label{tab:1}
\end{table}
\begin{table}[h!]
    \centering
    \begin{tabular}{|c|l|c|}
        \hline
        Last & Berechnung [\%] & Transaktionsdurchsatz [TXs/s] \\  \hline
        Last1 & 5 x Threads * 20 TXs/s & 100 \\ \hline
        Last2 & 2 x Client (5 x Threads * 20 TXs/s) & 200 \\ \hline
        Last3 & 3 x Client (5 x Threads * 20 TXs/s) & 300 \\ \hline
    \end{tabular}
    \caption{Tabelle für den Transaktiondurchsatz pro Sekunde bei den verschiedenen Last-Profilen}
    \label{tab:2}
\end{table}

Die drei folgenden Tabellen zeigen unsere Messungen, für die Transaktionen der Last 1, der Last 2 und der Last 3.
In den Tabellen sind die Anzahl der Gesamten Transaktionen, die Anzahl der Transaktionen pro sekunde für jede Relation und der Transaktionsdurchsatz insgesamt zu finden.
Außerdem ist jede Last auf allen 5 Versionen getestet worden, diese sind auch in der Tabelle vermerkt.
So kann man erkennen wie sich das Projekt entwickelt hat.

\subsection{Last1: Benchmark für 5 remote Load Driver}\label{subsec:benchmark-5-remote-load-driver}
\begin{table}[h!]
    \centering
    \begin{tabular}{|c|c|c|c|c|c|}
        \hline
        Version & Gesamte TXs & Kontostands-TXs & Analyse-TXs & Einzahlungs-TXs & TXs \\  \hline
        1 & 20.718 & 2.021 & 386 & 217 & 69 \\ \hline
        2 & 20.740 & 1.763 & 381 & 219 & 69 \\ \hline
        3 & 20.785 & 1.731 & 376 & 214 & 69 \\ \hline
        4 & 21.347 & 2.177 & 2.390 & 234 & 71 \\ \hline
        5 & 21.417 & 2.166 & 2.386 & 249 & 72 \\ \hline
    \end{tabular}
    \caption{Messergebnisse des Benchmarks bei Last1 (Transaktionen pro Sekunde)}
    \label{tab:3}
\end{table}

\newpage

\subsection{Last2: Benchmark für 5+5 remote Load Driver}\label{subsec:benchmark-5-5-remote-load-driver}
\begin{table}[h]
    \centering
    \begin{tabular}{|c|c|c|c|c|c|}
        \hline
        Version & Gesamte TXs & Kontostands-TXs & Analyse-TXs  & Einzahlungs-TXs & TXs \\  \hline
        1 & 39.737 & 2.545 & 412 & 418 & 132 \\ \hline
        2 & 39.777 & 2.524 & 407 & 413 & 133 \\ \hline
        3 & 39.912 & 2.614 & 414 & 420 & 133 \\ \hline
        4 & 42.694 & 4.355 & 4.780 & 469 & 142 \\ \hline
        5 & 42.835 & 4.333 & 4.773 & 498 & 143 \\ \hline
    \end{tabular}
    \caption{Messergebnisse des Benchmarks bei Last2 (Transaktionen pro Sekunde)}
    \label{tab:4}
\end{table}

\subsection{Last3: Benchmark für 5+5+5 remote Load Driver}\label{subsec:benchmark-5-5-5-remote-load-driver}
\begin{table}[h]
    \centering
        \begin{tabular}{|c|c|c|c|c|c|}
            \hline
            Version & Gesamte TXs & Kontostands--TXs & Analyse--TXs  & Einzahlungs--TXs & TXs \\
            \hline
            1 & 51.080 & 2.644 & 505 & 326 & 170 \\
            \hline
            2 & 54.824 & 2.103 & 438 & 468 & 183 \\
            \hline
            3 & 54.754 & 2.152 & 435 & 467 & 182 \\
            \hline
            4 & 63.997 & 6.514 & 7.197 & 735 & 213 \\
            \hline
            5 & 64.299 & 6.492 & 7.229 & 764 & 228 \\
            \hline
        \end{tabular}
        \caption{Messergebnisse des Benchmarks bei Last3 (Transaktionen pro Sekunde)}
        \label{tab:5}
\end{table}

[Last1--Version1: 69 \% zum maximalen erreichbaren Transaktionsdurchsatz (69 TXs/s)]
[Last1--Version5: 72 \% (72 TXs/s)] [$\Rightarrow$ 3 \% (3 TXs/s)]
[Last2--Version1: 66 \% (132 TXs/s)]
[Last2--Version5: 71 \% (143 TXs/s)] [$\Rightarrow$ 5,5 \% (11 TX/s)]
[Last3--Version1: 56 \% (170 TXs/s)]
[Last3--Version5: 76 \% (228 TXs/s)] [$\Rightarrow$ 19,4 \% (58 TX/s)]
[$\Rightarrow$ Steigerung des Transaktionsdurchsatzes der Versionen (exponentiell / proportional) durch Optimierungen]

\begin{figure}[h!]
    \center
    \begin{tikzpicture}[baseline]
        \begin{axis} [
            xlabel=Version,
            ylabel=$\frac{TX}{s}$,
            enlargelimits=false,
            axis on top,
            legend cell align=right,
            legend style={font=\footnotesize,legend pos=outer north east},
            ymajorgrids,
            xmajorgrids,
        ]

            \addplot[only marks, green] table [
                x={Version},
                y expr={\thisrow{TransactionsCount}/\thisrow{TransactionsTime}},
                col sep=comma,
                restrict expr to domain={\thisrow{Devices}}{1:1}
            ]
                {assets/values/messungen.csv};
            \addlegendentry{5 Threads}

            \addplot[only marks, blue] table [
                x={Version},
                y expr={\thisrow{TransactionsCount}/\thisrow{TransactionsTime}},
                col sep=comma,
                restrict expr to domain={\thisrow{Devices}}{2:2}
            ]
                {assets/values/messungen.csv};
            \addlegendentry{10 Threads}

            \addplot[only marks, violet] table [
                x={Version},
                y expr={\thisrow{TransactionsCount}/\thisrow{TransactionsTime}},
                col sep=comma,
                restrict expr to domain={\thisrow{Devices}}{3:3}
            ]
                {assets/values/messungen.csv};
            \addlegendentry{15 Threads}

        \end{axis}
    \end{tikzpicture}
    \caption{Transaktionen pro Thread im Verlauf der Versionen}
    \label{fig:tx-per-thread}
\end{figure}
    \begin{figure}[h!]
        \center
        \begin{tikzpicture}[baseline]
            \begin{axis} [
                xlabel=Version,
                ylabel={$\frac{TX}{s}$},
                enlargelimits=false,
                axis on top,
                legend cell align=right,
                legend style={font=\footnotesize,legend pos=outer north east},
                ymajorgrids,
                xmajorgrids,
            ]

                \addplot[only marks, green] table [
                x={Version},
                y expr={\thisrow{KontostandTXCount}/\thisrow{KontostandTXTime}},
                col sep=comma,
                restrict expr to domain={\thisrow{Devices}}{3:3}
                ]
                    {assets/values/messungen.csv};
                \addlegendentry{Kontostand}

                \addplot[only marks, blue] table [
                x={Version},
                y expr={\thisrow{AnalyseTXCount}/\thisrow{AnalyseTXTime}},
                col sep=comma,
                restrict expr to domain={\thisrow{Devices}}{3:3}
                ]
                    {assets/values/messungen.csv};
                \addlegendentry{Analyse}

                \addplot[only marks, violet] table [
                x={Version},
                y expr={\thisrow{EinzahlungsTXCount}/\thisrow{EinzahlungsTXTime}},
                col sep=comma,
                restrict expr to domain={\thisrow{Devices}}{3:3}
                ]
                    {assets/values/messungen.csv};
                \addlegendentry{Einzahlung}

            \end{axis}
        \end{tikzpicture}
        \caption{Transaktionsarten pro Sekunde im Verlauf der Versionen}
        \label{fig:tx-per-type}
    \end{figure}
