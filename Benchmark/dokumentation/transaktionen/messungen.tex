\subsection{Maximaler erreichbarer Transaktionsdurchsatz}\label{subsec:maximaler-erreichbarer-transaktionsdurchsatz}
Um den maximalen erreichbaren Transaktionsdurchsatz zu berechnen, müssen nur die begrenzenden Variablen, in diesem Fall Think-Time mit 50 ms, angeschaut werden.
Infolgedessen wird theoretisch angenommen, dass die verschiedenen Lasttransaktionen bei der Ausführung keine Zeit beanspruchen.
Schließlich ergibt sich dann aus dem Messzeitraum von 300s (5 min) und der begrenzenden Variable mit 0,05s der maximale erreichbare Transaktionsdurchsatz pro Thread im Loaddriver von 6.000 TXs (300s / 0,05s) und somit 20 TXs pro Sekunde.
Außerdem kann aus der relativen Gewichtung für die unterschiedlichen Lasttransaktionen folgende Verteilung der Transaktionsanzahl der verschiedenen Lasttransaktionen und Last-Profilen bestimmt werden.
\begin{table}[h!]
    \centering
    \begin{tabular}{|c|c|c|}
        \hline
        Lasttransaktion & relative Gewichtung [\%] & Transaktionsanzahl [TXs] \\  \hline
        Kontostands-TX & 35 & 2.100 \\ \hline
        Einzahlungs-TX & 50 & 3.000 \\ \hline
        Analyse-TX & 15 & 900 \\ \hline
    \end{tabular}
    \caption{Tabelle für die Verteilung der Transaktionsanzahl der unterschiedlichen Lasttransaktionen bei verschiedenen relativen Gewichtungen}
    \label{tab:1}
\end{table}
\begin{table}[h!]
    \centering
    \begin{tabular}{|c|l|c|}
        \hline
        Last & Berechnung [\%] & Transaktionsdurchsatz [TXs/s] \\  \hline
        Last1 & 5 x Threads * 20 TXs/s & 100 \\ \hline
        Last2 & 2 x Client (5 x Threads * 20 TXs/s) & 200 \\ \hline
        Last3 & 3 x Client (5 x Threads * 20 TXs/s) & 300 \\ \hline
    \end{tabular}
    \caption{Tabelle für den Transaktiondurchsatz pro Sekunde bei den verschiedenen Last-Profilen}
    \label{tab:2}
\end{table}

\subsection{Last1: Benchmark für 5 remote Load Driver}\label{subsec:benchmark-5-remote-load-driver}
\begin{table}[h!]
    \centering
    \begin{tabular}{|c|c|c|c|c|c|}
        \hline
        Version & Gesamte TXs & Kontostands-TXs/s & Analyse-TXs/s  & Einzahlungs-TXs/s & TXs/s \\  \hline
        1 & 20.718 & 2.021 & 386 & 217 & 69 \\ \hline
        2 & 20.740 & 1.763 & 381 & 219 & 69 \\ \hline
        3 & 20.785 & 1.731 & 376 & 214 & 69 \\ \hline
        4 & - & - & - & - & - \\ \hline
        5 & - & - & - & - & - \\ \hline
    \end{tabular}
    \caption{Messergebnisse des Benchmarks bei Last1}
    \label{tab:3}
\end{table}

\newpage

\subsection{Last2: Benchmark für 5+5 remote Load Driver}\label{subsec:benchmark-5-5-remote-load-driver}

[Log: transactions\_last2]
\begin{table}[h]
    \centering
    \begin{tabular}{|c|c|c|c|c|c|}
        \hline
        Version & Gesamte TXs & Kontostands-TXs/s & Analyse-TXs/s  & Einzahlungs-TXs/s & TXs/s \\  \hline
        1 & 39.737 & 2.545 & 412 & 418 & 132 \\ \hline
        2 & 39.777 & 2.524 & 407 & 413 & 133 \\ \hline
        3 & 39.912 & 2.614 & 414 & 420 & 133 \\ \hline
        4 & 42.694 & 4.355 & 4.780 & 469 & 142 \\ \hline
        5 & 42.835 & 4.333 & 4.773 & 498 & 143 \\ \hline
    \end{tabular}
    \caption{Messergebnisse des Benchmarks bei Last2 }
    \label{tab:4}
\end{table}

\subsection{Last3: Benchmark für 5+5+5 remote Load Driver}\label{subsec:benchmark-5-5-5-remote-load-driver}
[Log: transactions\_last3]
\begin{table}[h]
    \begin{center}
        \begin{tabular}{|c|c|c|c|c|c|}
            \hline
            \textbf{Version} & \textbf{Gesamte Transaktionen} &  \textbf{Kontostands-TXS
            s [TXs/s]} &  \textbf{Analyse-Txs [TXs/s]} &  \textbf{Einzahlungs-TXs [TXs/s]} &  \textbf{Transaktionen pro Sekunde}\\
            \hline\hline
            1 & 51.080 & 2.644 & 505 & 326 & 170 \\
            \hline
            2 & 54.824 & 2.103 & 438 & 468 & 183 \\
            \hline
            3 & 54.754 & 2.152 & 435 & 467 & 182 \\
            \hline
            4 & 63.997 & 6.514 & 7.197 & 735 & 213 \\
            \hline
            5 & 64.299 & 6.492 & 7.229 & 764 & 228 \\

        \end{tabular}
        \caption{Messungen Last 3}
        \label{Tabelle 3}
    \end{center}
\end{table}