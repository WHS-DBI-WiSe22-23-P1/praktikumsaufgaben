\subsection{Verbesserungsideen}\label{subsec:verbesserungsideen}
Nachfolgend werden mögliche Verbesserungsideen in Bezug auf den Programmcode und dem DBMS aufgelistet und mittels konkreter Implementierung bzw. Umsetzung erläutert.
\subsubsection{Batches}

\subsubsection{Views}
Als zweite Verbesserungsidee können Views in Betracht gezogen werden.
Diese sogenannten Database Views sind gespeicherte Abfragen, die beim Aufrufen eine Ergebnismenge erzeugen, wie eine virtuelle Tabelle.
Per Definition ist eine View eine benannte Abfrage, die im Datenbankkatalog gespeichert ist.

\subsection{Programmcode}\label{subsec:programmcode}
\subsubsection{Batches}

\subsubsection{Views}
Die Views wurden nur für die Kontostandtransaktion und der Analysetransaktion umgesetzt, da diese beiden ein SELECT-Statement ausführen, das sich für Database Views eignet.
Dabei muss vor der Nutzung der Views im Programmcode diese in der Datenbank hinterlegt werden, indem diese über die MySQL Workbench erstellt und an das DBMS gesendet werden.
\lstinputlisting[breaklines,language=SQL,label={lst:version2-createViewAccountsBalances},caption={SQL Statement zur Erstellung der View für die Kontostandtransaktion}]{assets/code/version2/createViewAccountsBalances.sql}
\lstinputlisting[breaklines,language=SQL,label={lst:version2-createViewAccountsBalanceNumbers},caption={SQL Statement zur Erstellung der View für die Analysetransaktion}]{assets/code/version2/createViewAccountsBalanceNumbers.sql}


\subsection{Datenbankmanagementsystem}\label{subsec:datenbankmanagementsystem}