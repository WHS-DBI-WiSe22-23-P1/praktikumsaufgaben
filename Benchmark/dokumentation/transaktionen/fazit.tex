Zusammenfassend kann man sagen, dass die vorgenommenen Optimierungen im Hinblick auf einen performanteren Code bzw. eines performaneteren DBMS den Transaktiondurchsatz deutlich erhöht haben.
Dies ist gut an den Messergebnissen der Benchmarks der jeweiligen Lasten und Programm-Versionen zu erkennen, da bei der ersten Programm-Version bei den unterschiedlichen Lasten der Transaktionsdurchsatz pro Sekunde nur zwischen 56,7 \% (Last3: 170 TXs/s) und 69 \% (Last1: 69 TXs/s) lag.
Nach verschiedenen Optimierungen am Programm-Code und am DMBS konnte der Transaktionsdurchsatz deutlich erhöht werden und lag bei der finalen Programm-Version zwischen 71.5 \% (Last2: 143 TXs/s) und 76 \% (Last3: 228 TX/s), dass wiederrum die Wirksamkeit der Optimierungen verdeutlicht.
